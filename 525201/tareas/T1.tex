\documentclass[12pt]{article}
\usepackage{amsmath, amssymb,amsmath,epsfig}
\usepackage[spanish,activeacute]{babel}   % para que salga todo en espaol en vez de ingles

%pre\'ambulo
\addtolength{\textwidth}{3cm}
\addtolength{\hoffset}{-2cm}
\addtolength{\voffset}{-2.5cm}
\addtolength{\textheight}{6cm}
\usepackage{amsmath,amsfonts}
\pagestyle{empty}

\font\rmm=cmr10 at 10truept       
\font\rmmm=cmr8 at 8truept
\font\bff=cmbx10 at 10truept
\font\lhf=cmdunh10 at 10truept
\font\lgf=manfnt at 10truept
\font\bl=cmss10 at 10truept
\font\sll=cmsl10 at 10truept
\font\itt=cmti10 at 10truept
\def\nt{\noindent}
\def\pp{\vspace{0.3cm}}
\def\pq{\vspace{0.5cm}}
\def\ph{\vspace{1.5cm}}

\def\R{\mathbb{R}}
\def\N{\mathbb{N}}
\def\Z{\mathbb{Z}}
\def\C{\mathbb{C}}
\def\Q{\mathbb{Q}}



\begin{document}

    \nt{\lhf UNIVERSIDAD DE CONCEPCION}\hfill Algebra III. 525201

    \nt {\bff FACULTAD DE CIENCIAS}\hfill Fecha de entrega: martes 13 de marzo. 

    \nt{\bff FISICAS Y MATEMATICAS}\hfill Hora de entrega: 12:00

    \nt {\bl  DEPARTAMENTO DE INGENIERIA MATEMATICA}\hfill Prof.: Anah\'i Gajardo


\pq

\hyphenation{re-cu-rren-cia}

\begin{center}
{\large Tarea 1}
\end{center}

\pq

Sean $p$, $q$, $r$, $s$ y $t$ cinco proposiciones cuyos valores de verdad son tales que la proposici\'on $[(p\vee q)\rightarrow (r\rightarrow s)]$ es falsa. Decida si esta informaci\'on permite o no determinar el valor de verdad de la siguiente proposici\'on:

\[\sim [ t \wedge \sim(p\wedge q)]\leftrightarrow (s\rightarrow r)\]

\vfill

\noindent\rule{16cm}{.5pt}\\
\noindent \today.\\
\noindent AGS/ags.


\end{document}
